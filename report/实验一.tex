\documentclass[12pt,a4paper]{article}
\usepackage[UTF8]{ctex}
\usepackage{amsmath}
\usepackage{amsfonts}
\usepackage{amssymb}
\usepackage{graphicx}
\usepackage{listings}
\usepackage{xcolor}
\usepackage{geometry}
\usepackage{booktabs}
\usepackage{array}
\usepackage{float}
\usepackage{longtable}

\geometry{left=2.5cm,right=2.5cm,top=2.5cm,bottom=2.5cm}

% 配置代码块样式(适配交换机命令行)
\lstset{
    language=bash,
    basicstyle=\ttfamily\small,
    keywordstyle=\color{blue},
    commentstyle=\color{gray!60},
    numbers=left,
    numberstyle=\tiny\color{gray},
    frame=tb,
    tabsize=4,
    showstringspaces=false,
    breaklines=true,
    escapeinside={/*@}{@*/},
    escapechar=|  % 用 | 作为转义符包裹 #
}

% 定义MIPS汇编语言
\lstdefinelanguage{MIPS}{
  comment=[l]{\#},
  keywords={add, addi, addiu, addu, and, andi, beq, bne, j, jal, jr, 
            lui, lw, or, ori, slt, slti, sltiu, sltu, sll, srl, 
            sub, subu, sw, xor, xori, syscall, break, li, move},
  keywordstyle=\color{blue}\bfseries,
  commentstyle=\color{green!60!black},
  stringstyle=\ttfamily,
  basicstyle=\small\ttfamily,
  breakatwhitespace=false,
  breaklines=true,
  captionpos=b,
  keepspaces=true,
  numbers=left,
  numbersep=5pt,
  numberstyle=\tiny,
  showspaces=false,
  showstringspaces=false,
  showtabs=false,
  tabsize=2
}

% 定义英文校名
\newcommand{\tongjiuniversityeng}{Tongji University}

% 自定义固定宽度列类型
\newcolumntype{L}[1]{>{\raggedright\let\newline\\\arraybackslash\hspace{0pt}}p{#1}}
\newcolumntype{C}[1]{>{\centering\let\newline\\\arraybackslash\hspace{0pt}}p{#1}}
\newcolumntype{R}[1]{>{\raggedleft\let\newline\\\arraybackslash\hspace{0pt}}p{#1}}

% 设置章节格式
\titleformat{\section}{\centering\Large\heiti}{第\chinese{section}部分}{1em}{}
\titleformat{\subsection}{\raggedright\large\heiti}{\chinese{subsection}.}{1em}{}

% 设置目录格式
\dottedcontents{section}[1.5em]{\bfseries}{1.5em}{1pc}
\dottedcontents{subsection}[3.5em]{\normalfont}{2.5em}{1pc}

% 设置封面
\newcommand{\MakeCover}[0]{
  \begin{titlepage}
    \begin{center}
      \vskip 60pt
      {\zihao{-1}\heiti 同济大学计算机科学与技术学院}
      \vskip 50pt
      {\zihao{-2}\heiti 计算机系统结构课程实验总结报告}
      \vskip 50pt

      % \includegraphics[width=0.35\textwidth]{tongji_logo.png}
      \vskip 50pt

      % 信息表格 - 使用固定宽度的列,内容居中
      \begin{tabular}{L{10em}@{\hspace{2em}}C{22em}}
        {\zihao{-2}\heiti 实验名称} & \uline{\makebox[22em]{简单的流水线CPU设计与性能分析}} \\
        \vspace{0.5cm} & \\
        {\zihao{-2}\heiti 学\quad{}\quad{}号} & \uline{\makebox[22em]{2351579}} \\
        \vspace{0.5cm} & \\
        {\zihao{-2}\heiti 姓\quad{}\quad{}名} & \uline{\makebox[22em]{程浩然}} \\
        \vspace{0.5cm} & \\
        {\zihao{-2}\heiti 专\quad{}\quad{}业} & \uline{\makebox[22em]{计算机科学与技术}} \\
        \vspace{0.5cm} & \\
        {\zihao{-2}\heiti 授课教师} & \uline{\makebox[22em]{秦国锋}} \\
        \vspace{0.5cm} & \\
        {\zihao{-2}\heiti 日\quad{}\quad{}期} & \uline{\makebox[22em]{\today}} \\
      \end{tabular}
      \vspace*{\fill}
    \end{center}
  \end{titlepage}
}

\begin{document}

\MakeCover

% 生成目录
\tableofcontents
\newpage

\section{实验环境部署与硬件配置说明}

本实验基于MIPS架构的5级流水线CPU设计。硬件平台采用Xilinx Vivado设计套件实现,主要包括:

\begin{itemize}
\item 流水线处理器核心:包括IF、ID、EX、MEM、WB五个阶段
\item 流水线寄存器:IF/ID、ID/EX、EX/MEM、MEM/WB寄存器
\item 存储器:指令存储器(IMEM)和数据存储器(DMEM)
\item 功能部件:算术逻辑单元(ALU)、寄存器文件、分支预测单元等
\end{itemize}

实验环境使用ModelSim进行仿真验证,通过run\_cpu\_tests.do脚本自动化运行测试用例。

\subsection{指令集实现}

指令集直接复用了计算机组成原理中的设计,采用如下的指令转义关系:

\begin{table}[h]
\centering
\begin{tabular}{|c|c|}
\hline
\textbf{指令} & \textbf{MIPS实现方式} \\
\hline
ADD & 采用add, addi, addiu, addu \\
\hline
NOP & 转义为sll \$0,\$0,0 \\
\hline
HALT & 创建了一条新的指令 \\
\hline
LOAD & lw \\
\hline
STORE & sw \\
\hline
CMP & sltu \$rd,\$rs,\$rt 或者 subu \$0,\$rs,\$rt \\
\hline
BZ & beq \$0,\$r,\$label \\
\hline
BN & bne \$0,\$r,\$label \\
\hline
\end{tabular}
\caption{指令转义关系表}
\end{table}

\subsection{分支指令设计}

分支指令设计以分支指令读取作为第0个周期:
\begin{itemize}
\item 在第0个周期下,IF\_ID应当直接输出PC\_bobl信号为1
\item 在第1个周期下,pc<=pc 同时 IMEM输出的指令应当是NOP
\end{itemize}

\section{实验的总体结构}

\subsection{5级指令流水线的总体结构}

本实验实现了MIPS架构的5级流水线处理器,包括以下5个阶段:

\begin{enumerate}
\item \textbf{IF (Instruction Fetch) - 指令获取阶段}:
  从指令存储器获取指令,更新程序计数器(PC)。

\item \textbf{ID (Instruction Decode) - 指令译码阶段}:
  译码指令,从寄存器文件读取操作数,检测分支指令。

\item \textbf{EX (Execute) - 执行阶段}:
  在ALU中执行操作,计算内存访问地址。

\item \textbf{MEM (Memory Access) - 内存访问阶段}:
  访问数据存储器,执行内存读/写操作。

\item \textbf{WB (Write Back) - 写回阶段}:
  将结果写回寄存器文件。
\end{enumerate}

\section{总体架构部件的解释说明}

\subsection{5级指令流水线总体结构部件的解释说明}

\subsubsection{IF/ID寄存器}
IF/ID寄存器连接指令获取阶段和指令译码阶段,保存从IF阶段传递到ID阶段的信息,包括:
\begin{itemize}
\item 指令本身
\item 当前PC值
\item 目标PC值
\end{itemize}

\subsubsection{ID/EX寄存器}
ID/EX寄存器连接指令译码阶段和执行阶段,包含:
\begin{itemize}
\item 译码后的指令
\item 操作数A和B
\item ALU控制信号
\end{itemize}

\subsubsection{EX/MEM寄存器}
EX/MEM寄存器连接执行阶段和内存访问阶段,包含:
\begin{itemize}
\item ALU计算结果
\item 内存访问控制信号
\item 内存写入数据
\end{itemize}

\subsubsection{MEM/WB寄存器}
MEM/WB寄存器连接内存访问阶段和写回阶段,包含:
\begin{itemize}
\item 内存访问结果
\item 写入寄存器的目标地址
\item 写使能信号
\end{itemize}

\section{实验仿真过程}

\subsection{5级指令流水线的仿真过程}

仿真过程使用ModelSim工具进行,主要包括以下步骤:

\begin{enumerate}
\item 编译所有Verilog源文件
\item 加载测试平台和待测CPU模块
\item 为不同测试用例加载相应的指令序列
\item 运行仿真并记录每个时钟周期的PC、指令和寄存器状态
\item 将仿真结果与软件参考模型进行比较验证
\end{enumerate}

自动化测试脚本run\_cpu\_tests.do会依次运行多个测试用例,包括ADDI、ADDIU、LW/SW、BEQ、BNE、SLL、SUBU、SLTU等指令的测试。

\subsection{启动测试与执行过程}

根据readme.md文件提供的信息,启动测试的指令为:
\begin{lstlisting}[language=bash]
vsim -c -do "do run_cpu_tests.do; quit" >log
\end{lstlisting}

该命令会在后台执行run\_cpu\_tests.do脚本并生成日志文件。

\subsection{分支预测与冲突处理机制}

\subsubsection{分支指令处理}
分支指令的处理机制如下:
\begin{itemize}
\item 以分支指令读取作为第0个周期
\item 在第0个周期下,IF\_ID应当直接输出PC\_bobl信号为1
\item 在第1个周期下,pc<=pc 同时 IMEM输出的指令应当是NOP
\end{itemize}

\subsubsection{冲突检测与处理}
\begin{enumerate}
\item 检测到冲突时,PC停止一个周期,向ID\_EX模块传送的指令变成NOP指令,冒泡一次
\item 冲突在ID段被检测出来,激活detect\_confict信号
\item PC、IMEM、IF\_ID保持原来持有的PC和instr一个周期
\end{enumerate}

\section{实验仿真的波形图及某时刻寄存器值的物理意义}

\subsection{5级指令流水线的波形图及某时刻寄存器值的物理意义}

在仿真过程中,每个时钟周期都会记录CPU的内部状态,包括:

\begin{itemize}
\item PC值:当前指令的程序计数器值
\item 指令:当前执行的指令
\item 寄存器文件:32个通用寄存器的值
\end{itemize}

寄存器值的物理意义:
\begin{itemize}
\item \$0:始终为0,是MIPS架构的固定约定
\item \$1-\$31:通用寄存器,用于存储操作数、结果和地址
\end{itemize}

通过分析波形图,可以观察到流水线的并行执行特性,在稳定状态时,每个时钟周期都有一个指令在不同阶段执行。

\section{流水线CPU实验性能验证模型}

\subsection{实验性能验证模型:比萨塔摔鸡蛋游戏}

设计C语言算法验证该模型:

% 配置代码块样式(适配MIPS汇编)
\lstset{
    language=C,
    basicstyle=\ttfamily\small,
    keywordstyle=\color{blue},
    commentstyle=\color{green!60!black},
    numbers=left,
    numberstyle=\tiny\color{gray},
    frame=tb,
    tabsize=4,
    showstringspaces=false,
    breaklines=true,
    escapeinside={/*@}{@*/},
    escapechar=|  % 用 | 作为转义符包裹 #
}

\begin{lstlisting}
// 比萨塔摔鸡蛋游戏验证模型的C算法
#include <stdio.h>

// 算法:使用二分搜索策略寻找鸡蛋的耐摔值
int egg_drop_simulation(int floors, int durability) {
    int drops = 0;         // 摔的总次数
    int eggs_used = 0;     // 使用的总鸡蛋数
    int broken_eggs = 0;   // 摔破的鸡蛋总数
    int current_floor = 1;
    int last_safe_floor = 0;
    int egg_broken = 0;    // 最后一个鸡蛋是否摔破

    // 首先使用二分搜索缩小范围
    int low = 1;
    int high = floors;
    int mid;

    while (low <= high) {
        drops++;
        eggs_used++;
        mid = (low + high) / 2;

        // 检查在mid层摔鸡蛋的情况
        if (mid <= durability) {
            // 鸡蛋没破,安全楼层提高
            last_safe_floor = mid;
            low = mid + 1;
        } else {
            // 鸡蛋破了,需要降低搜索范围
            broken_eggs++;
            high = mid - 1;
        }

        // 如果鸡蛋破了,我们需要线性搜索
        if (broken_eggs > 0) {
            // 从上次安全楼层+1开始逐层测试
            int test_floor = last_safe_floor + 1;
            while (test_floor < mid) {
                drops++;
                eggs_used++;
                if (test_floor > durability) {
                    // 鸡蛋破了
                    broken_eggs++;
                } else {
                    // 鸡蛋没破
                    last_safe_floor = test_floor;
                }
                test_floor++;
            }
            break;
        }
    }

    // 最后一个鸡蛋的状态
    if (broken_eggs > 0 && last_safe_floor + 1 > durability) {
        egg_broken = 1;
    } else {
        egg_broken = 0;
    }

    // 输出结果
    printf("摔的总次数: %d\n", drops);
    printf("使用的总鸡蛋数: %d\n", eggs_used);
    printf("摔破的鸡蛋总数: %d\n", broken_eggs);
    printf("最后摔的鸡蛋是否摔破: %s\n", egg_broken ? "是" : "否");
    
    return broken_eggs;  // 返回摔破的鸡蛋数
}

// 计算总成本f = m*p1 + n*p2 + h*p3
void calculate_cost(int floors_up, int floors_down, int broken_eggs, int p1, int p2, int p3) {
    int total_cost = floors_up * p1 + floors_down * p2 + broken_eggs * p3;
    printf("总成本 f = m*p1 + n*p2 + h*p3 = %d*%d + %d*%d + %d*%d = %d\n", 
           floors_up, p1, floors_down, p2, broken_eggs, p3, total_cost);
}

int main() {
    int durability = 25;  // 鸡蛋耐摔值(示例)
    int floors = 100;     // 比萨塔层数(示例)

    printf("比萨塔摔鸡蛋游戏验证模型\n");
    printf("鸡蛋耐摔值: %d, 比萨塔层数: %d\n\n", durability, floors);

    // 物质匮乏时期
    printf("物质匮乏时期 (p1=2, p2=1, p3=4):\n");
    int broken_eggs_1 = egg_drop_simulation(floors, durability);
    calculate_cost(50, 20, broken_eggs_1, 2, 1, 4);  // 示例值
    printf("\n");

    // 人力成本增长时期
    printf("人力成本增长时期 (p1=4, p2=1, p3=2):\n");
    int broken_eggs_2 = egg_drop_simulation(floors, durability);
    calculate_cost(30, 15, broken_eggs_2, 4, 1, 2);  // 示例值
    printf("\n");

    return 0;
}
\end{lstlisting}

\subsection{MIPS汇编程序}

将上述C算法转换为MIPS汇编程序:

% 配置代码块样式(适配MIPS汇编)
\lstset{
    language=MIPS,
    basicstyle=\ttfamily\small,
    keywordstyle=\color{blue},
    commentstyle=\color{green!60!black},
    numbers=left,
    numberstyle=\tiny\color{gray},
    frame=tb,
    tabsize=4,
    showstringspaces=false,
    breaklines=true,
    escapeinside={/*@}{@*/},
    escapechar=|  % 用 | 作为转义符包裹 #
}

\begin{lstlisting}[language=MIPS]
# 比萨塔摔鸡蛋游戏验证模型 - MIPS汇编实现
# 
# 使用寄存器约定:
# $a0 - 塔的总层数 (floors)
# $a1 - 鸡蛋耐摔值 (durability)
#
# $t0 - drops (摔的总次数)
# $t1 - eggs_used (使用的总鸡蛋数)
# $t2 - broken_eggs (摔破的鸡蛋总数)
# $t3 - current_floor (当前测试的楼层)
# $t4 - last_safe_floor (最后一个安全的楼层)
# $t5 - low (二分搜索的下界)
# $t6 - high (二分搜索的上界)
# $t7 - mid (二分搜索的中间值)

.text
.globl main

main:
    # 初始化值(示例)
    li $a0, 100        # 塔的总层数
    li $a1, 25         # 鸡蛋耐摔值

    # 跳转到egg_drop_simulation函数
    jal egg_drop_simulation

    # 准备计算成本
    li $a2, 50         # 楼上楼层总数 (示例值)
    li $a3, 20         # 楼下楼层总数 (示例值)
    
    # 物质匮乏时期 (p1=2, p2=1, p3=4)
    li $t8, 2          # p1
    li $t9, 1          # p2
    li $t10, 4         # p3
    jal calculate_cost
    move $s0, $v0      # 保存成本结果

    # 人力成本增长时期 (p1=4, p2=1, p3=2)
    li $t8, 4          # p1
    li $t9, 1          # p2
    li $t10, 2         # p3
    jal calculate_cost
    move $s1, $v0      # 保存成本结果

    # 程序结束
    li $v0, 10
    syscall

# 鸡蛋摔破模拟函数
# 输入: $a0 = 塔的总层数, $a1 = 鸡蛋耐摔值
# 输出: $v0 = 摔破的鸡蛋数
egg_drop_simulation:
    addi $sp, $sp, -20     # 分配栈空间
    sw $ra, 16($sp)        # 保存返回地址
    sw $s0, 12($sp)        # 保存其他寄存器
    sw $s1, 8($sp)
    sw $s2, 4($sp)
    sw $s3, 0($sp)

    # 初始化变量
    li $t0, 0              # drops = 0
    li $t1, 0              # eggs_used = 0
    li $t2, 0              # broken_eggs = 0
    li $t3, 1              # current_floor = 1
    li $t4, 0              # last_safe_floor = 0
    li $t5, 1              # low = 1
    move $t6, $a0          # high = floors

binary_search_loop:
    # 检查循环条件: low <= high
    bgt $t5, $t6, linear_search_loop
    
    # 计算 mid = (low + high) / 2
    add $t7, $t5, $t6
    srl $t7, $t7, 1        # 右移1位等于除2
    
    # 增加计数器
    addi $t0, $t0, 1       # drops++
    addi $t1, $t1, 1       # eggs_used++
    
    # 检查 mid <= durability (鸡蛋是否摔破)
    bgt $t7, $a1, egg_breaks
    
    # 鸡蛋没破: last_safe_floor = mid; low = mid + 1
    move $t4, $t7
    addi $t5, $t7, 1
    j binary_search_loop

egg_breaks:
    # 鸡蛋破了: broken_eggs++; high = mid - 1
    addi $t2, $t2, 1
    addi $t6, $t7, -1
    
    # 检查是否已经有过鸡蛋摔破
    bne $t2, $zero, linear_search_loop
    j binary_search_loop

linear_search_loop:
    # 线性搜索: 从 last_safe_floor+1 到 mid-1
    addi $t8, $t4, 1       # test_floor = last_safe_floor + 1
    move $t9, $t7          # test_floor <= mid-1

linear_search_iter:
    bge $t8, $t9, finished_search
    
    # 增加计数器
    addi $t0, $t0, 1       # drops++
    addi $t1, $t1, 1       # eggs_used++
    
    # 检查 test_floor > durability
    bgt $t8, $a1, linear_egg_breaks
    
    # 鸡蛋没破: last_safe_floor = test_floor
    move $t4, $t8
    addi $t8, $t8, 1       # test_floor++
    j linear_search_iter

linear_egg_breaks:
    # 鸡蛋破了: broken_eggs++
    addi $t2, $t2, 1
    addi $t8, $t8, 1       # test_floor++
    j linear_search_iter

finished_search:
    # 检查最后摔的鸡蛋是否摔破
    addi $t3, $t4, 1
    ble $t3, $a1, end_simulation
    
    # 最后一个鸡蛋摔破了
    # 这里可以设置标志但当前不需要

end_simulation:
    # 将结果放入返回寄存器
    move $v0, $t2          # 返回 broken_eggs
    
    # 恢复寄存器
    lw $s3, 0($sp)
    lw $s2, 4($sp)
    lw $s1, 8($sp)
    lw $s0, 12($sp)
    lw $ra, 16($sp)
    addi $sp, $sp, 20      # 释放栈空间
    
    jr $ra                 # 返回

# 计算总成本函数
# 输入: $a2 = 上楼层数 (floors_up)
#      $a3 = 下楼层数 (floors_down)
#      $v0 = 摔破鸡蛋数 (broken_eggs)
#      $t8 = p1, $t9 = p2, $t10 = p3
# 输出: $v0 = 总成本 f
calculate_cost:
    addi $sp, $sp, -16     # 分配栈空间
    sw $ra, 12($sp)        # 保存返回地址
    
    # 计算 f = floors_up * p1 + floors_down * p2 + broken_eggs * p3
    mult $a2, $t8          # floors_up * p1
    mflo $t0               # $t0 = floors_up * p1
    
    mult $a3, $t9          # floors_down * p2
    mflo $t1               # $t1 = floors_down * p2
    
    mult $v0, $t10         # broken_eggs * p3
    mflo $t2               # $t2 = broken_eggs * p3
    
    add $t0, $t0, $t1      # $t0 = floors_up * p1 + floors_down * p2
    add $v0, $t0, $t2      # $v0 = 总成本 f
    
    # 恢复寄存器
    lw $ra, 12($sp)
    addi $sp, $sp, 16      # 释放栈空间
    
    jr $ra                 # 返回
\end{lstlisting}

\section{实验验算程序下板测试过程与实现}

下板测试主要步骤包括:
\begin{enumerate}
\item 将设计下载到FPGA开发板
\item 配置测试向量和时钟源
\item 运行测试程序并监测输出结果
\item 验证功能正确性
\end{enumerate}

\section{流水线的性能指标定性分析}

\subsection{静态流水线的性能指标定性分析}

\subsubsection{吞吐率 (Throughput)}
理想情况下,5级流水线的吞吐率为1条指令/时钟周期,即在稳定状态下每个时钟周期可以完成一条指令。
但在实际应用中,由于数据相关、控制相关和结构冲突的存在,吞吐率会有所降低。

\subsubsection{加速比 (Speedup)}
理论上,5级流水线的理想加速比为5,但由于数据相关、控制相关和结构冲突,实际加速比会小于理论值。

\subsubsection{效率 (Efficiency)}
流水线效率 = 实际加速比 / 理论加速比,理想值为1,实际值小于1。
流水线效率受相关冲突、分支预测准确性等因素影响。

\subsubsection{相关与冲突分析}
\begin{itemize}
\item \textbf{数据相关}:通过数据前递技术解决,当无法前递时需要插入气泡(停顿周期)
\item \textbf{控制相关}:通过分支预测和延迟槽技术处理
\item \textbf{结构冲突}:通过增加硬件资源或插入停顿周期解决
\end{itemize}

\subsubsection{CPU的运行时间及存储器空间的使用}
流水线CPU在时间上通过并行处理多条指令提高了指令执行效率,在空间上需要额外的流水线寄存器来存储中间结果,但总体资源开销是可接受的。

\section{总结与体会}

通过对5级流水线CPU设计的学习和实践,我深刻理解了流水线技术在提高CPU性能方面的重要作用。通过将指令执行过程分解为多个阶段,并行处理多条指令,可以显著提高处理器的吞吐率。在设计过程中,需要处理各种相关和冲突问题,如数据相关、控制相关和结构冲突,这需要采用前递、分支预测、流水线停顿等技术来解决。

在本次实验中,我们实现了包括ADD、ADDU、ADDI、ADDIU、SLL、LW、SW、SUBU、BNE、BEQ、SLTU和HALT在内的12条指令的流水线处理器。通过数据前递技术解决了数据相关问题,通过分支预测机制处理控制相关问题。

流水线技术虽然提高了指令执行效率,但也引入了复杂性,如数据冒险、控制冒险和结构冒险的处理。通过本次实验,我不仅掌握了流水线的基本概念和设计方法,还学会了如何分析和解决流水线中的各种冲突问题,对计算机系统结构有了更深入的理解。

\section{附件(所有程序)}

\subsection{5级指令流水线的设计程序}

\subsubsection{顶层模块 (sccomp\_dataflow.v)}
\lstinputlisting[language=Verilog]{../cpupip8.srcs/sources_1/new/sccomp_dataflow.v}

\subsubsection{CPU核心模块 (cpu.v)}
\lstinputlisting[language=Verilog]{../cpupip8.srcs/sources_1/new/cpu.v}

\subsubsection{定义文件 (def.v)}
\lstinputlisting[language=Verilog]{../cpupip8.srcs/sources_1/new/def.v}

\subsubsection{算术逻辑单元 (alu.v)}
\lstinputlisting[language=Verilog]{../cpupip8.srcs/sources_1/new/alu.v}

\subsubsection{分支判断模块 (BJudge.v)}
\lstinputlisting[language=Verilog]{../cpupip8.srcs/sources_1/new/BJudge.v}

\subsubsection{PC寄存器模块 (PCreg.v)}
\lstinputlisting[language=Verilog]{../cpupip8.srcs/sources_1/new/PCreg.v}

\subsubsection{数据存储器 (DMEM.v)}
\lstinputlisting[language=Verilog]{../cpupip8.srcs/sources_1/new/DMEM.v}

\subsubsection{流水级间寄存器 (EX\_MEM.v)}
\lstinputlisting[language=Verilog]{../cpupip8.srcs/sources_1/new/EX_MEM.v}

\subsubsection{流水级间寄存器 (ID\_EX.v)}
\lstinputlisting[language=Verilog]{../cpupip8.srcs/sources_1/new/ID_EX.v}

\subsubsection{流水级间寄存器 (IF\_ID.v)}
\lstinputlisting[language=Verilog]{../cpupip8.srcs/sources_1/new/IF_ID.v}

\subsubsection{指令存储器 (IMEM.v)}
\lstinputlisting[language=Verilog]{../cpupip8.srcs/sources_1/new/IMEM.v}

\subsubsection{流水级间寄存器 (MEM\_WB.v)}
\lstinputlisting[language=Verilog]{../cpupip8.srcs/sources_1/new/MEM_WB.v}

\subsubsection{下一PC生成模块 (NPCmaker.v)}
\lstinputlisting[language=Verilog]{../cpupip8.srcs/sources_1/new/NPCmaker.v}

\subsubsection{寄存器文件 (regfile.v)}
\lstinputlisting[language=Verilog]{../cpupip8.srcs/sources_1/new/regfile.v}

\subsubsection{测试平台 (\_246tb\_ex10\_tb.v)}
\lstinputlisting[language=Verilog]{../cpupip8.srcs/sim_1/new/_246tb_ex10_tb.v}

\subsubsection{自动化测试脚本 (run\_cpu\_tests.do)}
\lstinputlisting{../run_cpu_tests.do}

\subsubsection{软件仿真器 (cpu\_pipelined\_simulator.cpp)}
\lstinputlisting[language=C++]{../cpu_pipelined_simulator.cpp}

\subsection{写锁设计机制}

寄存器文件的写锁设计是解决数据冲突的关键机制:

\begin{enumerate}
\item 在指令流过regfile的时候为将会写的寄存器上写锁。等到写回阶段,释放写锁。
\item 写与写之间一定不会冲突,因为后来的写一定在先来的写之后。
\item 读与写之间有可能会冲突,这个冲突会在ALU或者DMEM的位置重定向。
\item 为每一个写锁设置一个定时器,定时器时长总为3,因为只有在3个周期之后才能等回这个地方的数据。定时器归0时释放写锁。
\item 如果冲突无法通过重定向解决,则需要等待,这个时候PC停止一个周期,向ID\_EX模块传送的指令变成NOP指令,冒泡一次。
\item reglock低2位为计时器,前2位标志是ALU型占用还是DMEM型占用。
\item 这个冲突在ID段被检测出来,激活detact\_confict信号,PC、IMEM、IF\_ID保持原来持有的PC和instr一个周期。
\end{enumerate}

\end{document}