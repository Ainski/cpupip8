\documentclass[12pt,a4paper]{article}
% 调整包加载顺序:ctex优先,避免冲突
\usepackage[UTF8]{ctex}
% 核心依赖包
\usepackage{amsmath,amsfonts,amssymb}
\usepackage{graphicx}
\usepackage{geometry}
\usepackage{booktabs}
\usepackage{array}
\usepackage{float}
\usepackage{longtable}
\usepackage{titlesec}
% 使用更安全的代码引入方式
\usepackage{verbatim}
\usepackage{multirow}

% 页面布局
\geometry{left=2.5cm,right=2.5cm,top=2.5cm,bottom=2.5cm}

% 自定义列类型
\newcolumntype{L}[1]{>{\raggedright\let\newline\\\arraybackslash\hspace{0pt}}p{#1}}
\newcolumntype{C}[1]{>{\centering\let\newline\\\arraybackslash\hspace{0pt}}p{#1}}
\newcolumntype{R}[1]{>{\raggedleft\let\newline\\\arraybackslash\hspace{0pt}}p{#1}}

% 章节格式
\titleformat{\section}{\centering\Large\heiti}{第\chinese{section}部分}{1em}{}
\titleformat{\subsection}{\raggedright\large\heiti}{\chinese{subsection}.}{1em}{}

% 定义英文校名
\newcommand{\tongjiuniversityeng}{Tongji University}

% 修复封面命令
\newcommand{\MakeCover}{
  \begin{titlepage}
    \begin{center}
      \vskip 60pt
      {\fontsize{22pt}{26pt}\bfseries 同济大学计算机科学与技术学院}
      \vskip 50pt
      {\fontsize{18pt}{22pt}\bfseries 计算机系统结构课程实验总结报告}
      \vskip 50pt

      % 校徽:注释或替换为实际路径
      \includegraphics[width=0.35\textwidth]{tongji_logo.png}
      \vskip 50pt

      % 封面表格
      \begin{tabular}{L{8em}@{\hspace{1em}}C{18em}}
        {\fontsize{18pt}{22pt}\bfseries 实验名称} & \underline{\makebox[18em]{简单的流水线CPU设计与性能分析}} \\[0.8cm]
        {\fontsize{18pt}{22pt}\bfseries 学号} & \underline{\makebox[18em]{2351579}} \\[0.8cm]
        {\fontsize{18pt}{22pt}\bfseries 姓名} & \underline{\makebox[18em]{程浩然}} \\[0.8cm]
        {\fontsize{18pt}{22pt}\bfseries 专业} & \underline{\makebox[18em]{计算机科学与技术}} \\[0.8cm]
        {\fontsize{18pt}{22pt}\bfseries 授课教师} & \underline{\makebox[18em]{秦国锋}} \\[0.8cm]
        {\fontsize{18pt}{22pt}\bfseries 日期} & \underline{\makebox[18em]{\today}} \\
      \end{tabular}
      \vspace*{\fill}
    \end{center}
  \end{titlepage}
}

\begin{document}

% 生成封面
\MakeCover

% 生成目录
\tableofcontents
\newpage

\section{实验环境部署与硬件配置说明}

本实验围绕MIPS架构的5级流水线CPU展开设计,硬件平台基于Xilinx Vivado设计套件实现,核心硬件组成包括:

\begin{itemize}
\item 流水线处理器核心:划分为取指(IF)、译码(ID)、执行(EX)、访存(MEM)、写回(WB)五个阶段
\item 流水线寄存器:包含IF/ID、ID/EX、EX/MEM、MEM/WB四级流水寄存器
\item 存储模块:指令存储器(IMEM)与数据存储器(DMEM)
\item 功能部件:算术逻辑单元(ALU)、寄存器文件、分支预测单元等
\end{itemize}

实验采用ModelSim进行仿真验证,通过自动化测试脚本run\_cpu\_tests.do批量运行测试用例,完成功能与性能的验证。

\subsection{指令集实现}

本实验复用计算机组成原理课程中的指令集设计方案,指令与MIPS架构的映射关系如下表所示:

\begin{table}[h]
\centering
\begin{tabular}{|c|c|}
\hline
\textbf{指令} & \textbf{MIPS实现方式} \\
\hline
ADD & 采用add、addi、addiu、addu指令实现 \\
\hline
NOP & 转义为sll \$0,\$0,0指令 \\
\hline
HALT & 自定义新增指令实现 \\
\hline
LOAD & 复用lw指令实现 \\
\hline
STORE & 复用sw指令实现 \\
\hline
CMP & 通过sltu \$rd,\$rs,\$rt或subu \$0,\$rs,\$rt指令实现 \\
\hline
BZ & 转义为beq \$0,\$r,\$label指令 \\
\hline
BN & 转义为bne \$0,\$r,\$label指令 \\
\hline
\end{tabular}
\caption{指令转义关系表}
\end{table}

设计中手绘cpu架构图如下
\begin{figure}[!htp]
\centering
\includegraphics[width=0.7\linewidth]{screenshot003}
\caption{}
\label{fig:screenshot003}
\end{figure}

\clearpage
\subsection{分支指令设计}

分支指令的执行周期以指令读取为第0周期,具体处理逻辑为:
\begin{itemize}
\item 第0周期:IF\_ID寄存器直接输出PC\_bobl信号为1,触发分支检测
\item 第1周期:程序计数器PC保持当前值不变,同时指令存储器IMEM输出NOP指令,插入流水线气泡
\end{itemize}

\section{实验的总体结构}

\subsection{5级指令流水线的总体结构}

本实验实现的MIPS架构5级流水线处理器,将指令执行过程拆解为以下五个阶段,各阶段的核心功能如下:

\begin{enumerate}
\item \textbf{取指阶段(IF, Instruction Fetch)}:
  从指令存储器中读取当前PC指向的指令,并完成程序计数器的自增更新。

\item \textbf{译码阶段(ID, Instruction Decode)}:
  对取指阶段获取的指令进行译码,从寄存器文件中读取操作数,并检测分支指令的跳转条件。

\item \textbf{执行阶段(EX, Execute)}:
  在算术逻辑单元ALU中执行译码后的运算操作,同时计算内存访问的有效地址。

\item \textbf{访存阶段(MEM, Memory Access)}:
  根据执行阶段的结果,完成数据存储器的读/写操作,实现数据的访存交互。

\item \textbf{写回阶段(WB, Write Back)}:
  将执行结果或访存数据写回寄存器文件,完成指令的最终执行流程。
\end{enumerate}

\section{总体架构部件的解释说明}

\subsection{5级指令流水线总体结构部件的解释说明}

\subsubsection{IF/ID寄存器}
IF/ID寄存器作为取指与译码阶段的衔接部件,主要用于暂存从IF阶段传递至ID阶段的关键信息,具体包括:
\begin{itemize}
\item 取指阶段获取的指令编码
\item 当前程序计数器PC的数值
\item 分支指令的目标PC地址
\end{itemize}

\subsubsection{ID/EX寄存器}
ID/EX寄存器承担译码与执行阶段的信息传递功能,暂存的核心数据包括:
\begin{itemize}
\item 译码后的指令操作码与操作数信息
\item 从寄存器文件中读取的操作数A和操作数B
\item 控制单元生成的ALU运算控制信号
\end{itemize}

\subsubsection{EX/MEM寄存器}
EX/MEM寄存器连接执行与访存阶段,主要存储执行阶段的运算结果与访存控制信息,包括:
\begin{itemize}
\item ALU的运算结果与内存访问地址
\item 数据存储器的读/写控制信号
\item 待写入数据存储器的原始数据
\end{itemize}

\subsubsection{MEM/WB寄存器}
MEM/WB寄存器是访存与写回阶段的桥梁,暂存的信息用于完成最终的写回操作,包括:
\begin{itemize}
\item 数据存储器的读取结果
\item 待写入寄存器文件的目标寄存器地址
\item 寄存器写回的使能控制信号
\end{itemize}

\section{实验仿真过程}

\subsection{5级指令流水线的仿真过程}

本实验采用ModelSim工具完成流水线CPU的仿真验证,具体仿真流程分为以下步骤:

\begin{enumerate}
\item 编译所有Verilog硬件描述语言源文件,生成可仿真的模块库
\item 加载测试平台模块与待测CPU核心模块,建立仿真拓扑
\item 为不同测试场景加载对应的指令序列,配置仿真激励
\item 运行仿真流程,实时记录每个时钟周期的PC值、指令内容与寄存器状态
\item 将仿真输出结果与软件参考模型的计算结果对比,验证功能正确性
\end{enumerate}

自动化测试脚本run\_cpu\_tests.do会按序执行多组测试用例,覆盖ADDI、ADDIU、LW/SW、BEQ、BNE、SLL、SUBU、SLTU等核心指令的功能验证。

\subsection{启动测试与执行过程}

根据实验配套的readme.md文档说明,通过以下指令启动仿真测试:
\begin{verbatim}
vsim -c -do "do run_cpu_tests.do; quit" >log
\end{verbatim}

该命令以批处理模式运行ModelSim,自动执行run\_cpu\_tests.do脚本并将仿真日志输出至log文件,便于后续结果分析。

\subsection{分支预测与冲突处理机制}

\subsubsection{分支指令处理}
分支指令的流水线处理机制以指令读取为第0周期,具体逻辑为:
\begin{itemize}
\item 第0周期:IF\_ID寄存器直接输出PC\_bobl信号为1,触发分支检测逻辑
\item 第1周期:程序计数器PC保持当前值,指令存储器IMEM输出NOP指令,插入流水线气泡
\end{itemize}

\subsubsection{冲突检测与处理}
流水线冲突的检测与处理流程如下:
\begin{enumerate}
\item 当检测到流水线冲突时,PC暂停一个时钟周期更新,向ID\_EX模块传递NOP指令,完成一次流水线冒泡
\item 冲突检测在译码(ID)阶段完成,检测到冲突后激活detect\_confict信号
\item PC、指令存储器IMEM与IF/ID寄存器在冲突周期内保持原有PC值与指令内容,确保流水线稳定
\end{enumerate}

\subsection{写锁设计机制}

寄存器文件的写锁设计是解决流水线数据冲突的核心机制,具体实现逻辑如下:

\begin{enumerate}
\item 当指令流经寄存器文件时,为待写入的寄存器添加写锁,直至写回阶段完成后释放写锁
\item 写操作之间不存在冲突,因为后发的写指令必然在先行的写指令之后执行
\item 读操作与写操作可能产生冲突,此类冲突需在ALU或DMEM模块中通过数据重定向解决
\item 为每个写锁配置时长为3的定时器,定时器归零时自动释放写锁(因数据需3个周期完成传递)
\item 若冲突无法通过重定向解决,则暂停PC更新一个周期,向ID\_EX模块传递NOP指令,完成一次流水线冒泡
\item 写锁寄存器reglock的低2位为计时器,高2位标识锁的类型(ALU型占用或DMEM型占用)
\item 冲突检测在译码(ID)阶段完成,检测到冲突后激活detact\_confict信号,PC、IMEM与IF/ID寄存器保持原有值一个周期
\end{enumerate}

\section{实验仿真的波形图及某时刻寄存器值的物理意义}

\subsection{5级指令流水线的波形图及某时刻寄存器值的物理意义}

在仿真过程中,ModelSim会实时记录CPU内部的关键信号波形,核心监测内容包括:

\begin{itemize}
\item PC值:当前正在执行指令的程序计数器地址,反映指令的执行流
\item 指令编码:当前流水线各阶段的指令内容,体现指令的执行进度
\item 寄存器文件:32个通用寄存器的实时数值,反映数据的运算与传递过程
\end{itemize}

MIPS架构中寄存器值的物理意义明确:
\begin{itemize}
\item \$0寄存器:硬件层面固定为0,是MIPS架构的固有约定,用于零值传递与空操作
\item \$1-\$31寄存器:通用寄存器,用于存储运算操作数、中间结果与内存访问地址
\end{itemize}

通过分析仿真波形图,可直观观察到流水线的并行执行特性:在稳定运行状态下,每个时钟周期均有一条指令在不同流水线阶段执行,实现指令的流水化处理。

\section{流水线CPU实验性能验证模型}

\subsection{实验性能验证模型:比萨塔摔鸡蛋游戏}

为验证流水线CPU的性能,设计如下MIPS语言算法实现比萨塔摔鸡蛋游戏的仿真模型:

将上述C语言算法转换为MIPS汇编程序,实现流水线CPU的性能验证,代码如下:

\begin{verbatim}
addiu $s0, $zero, 100    # building_height = 100
addiu $s1, $zero, 42     # egg_durability = 42

# 物质匮乏时期成本参数
addiu $s2, $zero, 2      # scarcity_p1 = 2
addiu $s3, $zero, 1      # scarcity_p2 = 1  
addiu $s4, $zero, 4      # scarcity_p3 = 4

# 人力成本增长时期成本参数
addiu $s5, $zero, 4      # labor_p1 = 4
addiu $s6, $zero, 1      # labor_p2 = 1
addiu $s7, $zero, 2      # labor_p3 = 2

# 开始模拟物质匮乏时期
# 初始化游戏变量
addiu $t0, $zero, 0      # total_drops = 0
addiu $t1, $zero, 0      # total_eggs_broken = 0
addiu $t2, $zero, 0      # last_egg_broken = 0
addiu $t3, $zero, 0      # total_cost = 0
addiu $t4, $zero, 1      # low = 1
addiu $t5, $zero, 100    # high = building_height
addiu $t6, $zero, 0      # current_floor = 0
addiu $t7, $zero, 2      # eggs_remaining = 2
addiu $t8, $zero, 0      # up_floors = 0
addiu $t9, $zero, 0      # down_floors = 0

# 物质匮乏时期游戏循环
scarcity_loop:
# 检查循环条件: low <= high && eggs_remaining > 0
sltu $at, $t5, $t4       # high < low?
bne $at, $zero, scarcity_end
beq $t7, $zero, scarcity_end

# 计算mid = (low + high) / 2
addu $at, $t4, $t5
srl $at, $at, 1

# 计算上下楼
sltu $k0, $t6, $at       # current_floor < mid?
bne $k0, $zero, scarcity_go_up

# 下楼情况
subu $k0, $t6, $at
addu $t9, $t9, $k0
beq $zero, $zero, scarcity_update_pos

scarcity_go_up:
# 上楼情况
subu $k0, $at, $t6
addu $t8, $t8, $k0

scarcity_update_pos:
# current_floor = mid
addiu $t6, $at, 0

# total_drops++
addiu $t0, $t0, 1

# 检查鸡蛋是否摔破
sltu $k0, $s1, $at       # egg_durability < mid?
bne $k0, $zero, scarcity_egg_broken

# 鸡蛋没破
addiu $t2, $zero, 0      # last_egg_broken = 0
addiu $t4, $at, 1        # low = mid + 1
beq $zero, $zero, scarcity_continue

scarcity_egg_broken:
# 鸡蛋摔破
addiu $t1, $t1, 1        # total_eggs_broken++
addiu $t7, $t7, -1       # eggs_remaining--
addiu $t2, $zero, 1      # last_egg_broken = 1
addiu $t5, $at, -1       # high = mid - 1

scarcity_continue:
beq $zero, $zero, scarcity_loop

scarcity_end:
# 计算物质匮乏时期总成本: up_floors*p1 + down_floors*p2 + eggs_broken*p3
addiu $k0, $zero, 0      # temp_cost = 0

# 计算 up_floors * p1 (加法模拟乘法)
addiu $k1, $zero, 0      # 乘法计数器
scarcity_mult1_loop:
beq $t8, $zero, scarcity_mult1_done
addu $k0, $k0, $s2
addiu $t8, $t8, -1
beq $zero, $zero, scarcity_mult1_loop
scarcity_mult1_done:

# 计算 down_floors * p2
addiu $k1, $zero, 0
scarcity_mult2_loop:
beq $t9, $zero, scarcity_mult2_done
addu $k0, $k0, $s3
addiu $t9, $t9, -1
beq $zero, $zero, scarcity_mult2_loop
scarcity_mult2_done:

# 计算 eggs_broken * p3
addiu $k1, $zero, 0
scarcity_mult3_loop:
beq $t1, $zero, scarcity_mult3_done
addu $k0, $k0, $s4
addiu $t1, $t1, -1
beq $zero, $zero, scarcity_mult3_loop
scarcity_mult3_done:

# 保存物质匮乏时期结果
addiu $t8, $t0, 0        # total_drops -> $t8
addiu $t9, $t1, 0        # total_eggs_broken -> $t9
addiu $k0, $t2, 0        # last_egg_broken -> $k0
addiu $k1, $k0, 0        # total_cost -> $k1

# 开始模拟人力成本增长时期
# 重新初始化游戏变量
addiu $t0, $zero, 0      # total_drops = 0
addiu $t1, $zero, 0      # total_eggs_broken = 0
addiu $t2, $zero, 0      # last_egg_broken = 0
addiu $t3, $zero, 0      # total_cost = 0
addiu $t4, $zero, 1      # low = 1
addiu $t5, $zero, 100    # high = building_height
addiu $t6, $zero, 0      # current_floor = 0
addiu $t7, $zero, 2      # eggs_remaining = 2
addiu $gp, $zero, 0      # up_floors = 0
addiu $sp, $zero, 0      # down_floors = 0

# 人力成本增长时期游戏循环
labor_loop:
# 检查循环条件
sltu $at, $t5, $t4       # high < low?
bne $at, $zero, labor_end
beq $t7, $zero, labor_end

# 计算mid = (low + high) / 2
addu $at, $t4, $t5
srl $at, $at, 1

# 计算上下楼
sltu $fp, $t6, $at       # current_floor < mid?
bne $fp, $zero, labor_go_up

# 下楼情况
subu $fp, $t6, $at
addu $sp, $sp, $fp
beq $zero, $zero, labor_update_pos

labor_go_up:
# 上楼情况
subu $fp, $at, $t6
addu $gp, $gp, $fp

labor_update_pos:
# current_floor = mid
addiu $t6, $at, 0

# total_drops++
addiu $t0, $t0, 1

# 检查鸡蛋是否摔破
sltu $fp, $s1, $at       # egg_durability < mid?
bne $fp, $zero, labor_egg_broken

# 鸡蛋没破
addiu $t2, $zero, 0      # last_egg_broken = 0
addiu $t4, $at, 1        # low = mid + 1
beq $zero, $zero, labor_continue

labor_egg_broken:
# 鸡蛋摔破
addiu $t1, $t1, 1        # total_eggs_broken++
addiu $t7, $t7, -1       # eggs_remaining--
addiu $t2, $zero, 1      # last_egg_broken = 1
addiu $t5, $at, -1       # high = mid - 1

labor_continue:
beq $zero, $zero, labor_loop

labor_end:
# 计算人力成本增长时期总成本
addiu $ra, $zero, 0      # temp_cost = 0

# 计算 up_floors * p1
addiu $fp, $zero, 0
labor_mult1_loop:
beq $gp, $zero, labor_mult1_done
addu $ra, $ra, $s5
addiu $gp, $gp, -1
beq $zero, $zero, labor_mult1_loop
labor_mult1_done:

# 计算 down_floors * p2
addiu $fp, $zero, 0
labor_mult2_loop:
beq $sp, $zero, labor_mult2_done
addu $ra, $ra, $s6
addiu $sp, $sp, -1
beq $zero, $zero, labor_mult2_loop
labor_mult2_done:

# 计算 eggs_broken * p3
addiu $fp, $zero, 0
labor_mult3_loop:
beq $t1, $zero, labor_mult3_done
addu $ra, $ra, $s7
addiu $t1, $t1, -1
beq $zero, $zero, labor_mult3_loop
labor_mult3_done:

# 保存人力成本增长时期结果
addiu $gp, $t0, 0        # total_drops -> $gp
addiu $sp, $t1, 0        # total_eggs_broken -> $sp
addiu $fp, $t2, 0        # last_egg_broken -> $fp
addiu $ra, $ra, 0        # total_cost -> $ra

# 程序结束
halt
\end{verbatim}
\subsection{比萨塔摔鸡蛋游戏验证模型结果分析}

根据比萨塔摔鸡蛋游戏的模拟结果,使用二分查找策略在100层建筑中测试鸡蛋耐摔值(42层),得到以下计算结果:

\subsubsection{游戏模拟参数}
\begin{itemize}
    \item 建筑高度:100层
    \item 鸡蛋耐摔值:42层
    \item 测试策略:二分查找算法
    \item 初始鸡蛋数:2个
\end{itemize}

\subsubsection{模拟过程详细分析}
二分查找测试序列:
\begin{enumerate}
    \item \textbf{测试50层}:鸡蛋摔破($50 > 42$),上楼50层,摔破1个鸡蛋
    \item \textbf{测试25层}:鸡蛋未破($25 \leq 42$),下楼25层
    \item \textbf{测试37层}:鸡蛋未破($37 \leq 42$),上楼12层
    \item \textbf{测试43层}:鸡蛋摔破($43 > 42$),上楼6层,摔破1个鸡蛋
\end{enumerate}

\subsubsection{MIPS寄存器存储结果}
程序执行完毕后,结果存储在以下寄存器中:

\begin{tabular}{lll}
\hline
\textbf{时期} & \textbf{寄存器} & \textbf{数值和含义} \\
\hline
\multirow{4}{*}{物质匮乏时期} 
& \$20 & 4(摔的总次数) \\
& \$18 & 2(摔破的鸡蛋总数) \\
& \$19 & 1(最后鸡蛋状态:摔破) \\
& \$14 & 101 = 0x65(总成本 14 号寄存器值) \\
\hline
\multirow{4}{*}{人力成本增长时期}
& \$28 & 4(摔的总次数) \\
& \$29 & 2(摔破的鸡蛋总数) \\
& \$30 & 1(最后鸡蛋状态:摔破) \\
& \$31 & 306 = 0x198(总成本 31 号寄存器值) \\
\hline
\end{tabular}
\begin{figure}[!htp]
\centering
\includegraphics[width=0.7\linewidth]{screenshot001}
\caption{}
\label{fig:screenshot001}
\end{figure}


\subsubsection{成本分析结论}
人力成本增长时期的总成本(301)显著高于物质匮乏时期(169),主要原因是上楼成本$p_1$从2增加到4,而鸡蛋成本$p_3$的降低不足以抵消这一影响,反映了不同历史时期资源成本的变迁。


\section{实验验算程序下板测试过程与实现}

流水线CPU的FPGA下板测试遵循以下步骤完成:
\begin{enumerate}
\item 将Verilog设计文件综合后,通过JTAG下载至FPGA开发板
\item 配置测试向量生成模块与时钟源,设置仿真激励参数
\item 运行测试程序,通过逻辑分析仪监测CPU的输出结果与内部信号
\item 对比FPGA实测结果与ModelSim仿真结果,验证硬件功能的正确性
\end{enumerate}

\begin{figure}[!htp]
\centering
\includegraphics[width=0.7\linewidth]{screenshot002}
\caption{}
\label{fig:screenshot002}
\end{figure}


\section{流水线的性能指标定性分析}

\subsection{静态流水线的性能指标定性分析}

\subsubsection{吞吐率 (Throughput)}
理想状态下,5级流水线CPU的吞吐率为1条指令/时钟周期,即流水线稳定运行时,每个时钟周期可完成一条指令的执行。但在实际场景中,数据相关、控制相关与结构冲突会导致流水线停顿,使得实际吞吐率低于理想值。

\subsubsection{加速比 (Speedup)}
理论上,5级流水线的理想加速比为5,即相比单周期CPU,5级流水线的指令执行效率可提升5倍。但由于流水线冲突与停顿的存在,实际加速比会显著低于理论值。

\subsubsection{效率 (Efficiency)}
流水线效率的计算公式为:实际加速比 / 理论加速比,理想效率为1。实际效率受数据相关、控制相关、分支预测准确性等因素影响,通常小于1。

\subsubsection{相关与冲突分析}
流水线执行过程中的核心冲突类型及解决策略如下:
\begin{itemize}
\item \textbf{数据相关}:通过数据前递技术直接传递运算结果,无法前递时插入流水线气泡(停顿周期)
\item \textbf{控制相关}:采用分支预测与延迟槽技术,减少分支指令带来的流水线冲刷
\item \textbf{结构冲突}:通过增加硬件资源(如多端口存储器)或插入停顿周期,避免功能部件的资源竞争
\end{itemize}

\subsubsection{CPU的运行时间及存储器空间的使用}
流水线CPU通过指令的并行执行,大幅缩短了指令的平均执行时间,提升了整体运行效率;在存储资源方面,流水线需要额外的流水寄存器暂存中间结果,虽增加了少量硬件开销,但整体资源占用处于可接受范围,是性能与资源的合理权衡。

\section{总结与体会}

通过本次MIPS架构5级流水线CPU的设计与实现实验,我深入理解了流水线技术在提升CPU性能中的核心作用。将指令执行过程拆解为取指、译码、执行、访存、写回五个阶段,通过并行处理多条指令,能够显著提高处理器的指令吞吐率。在设计过程中,针对数据相关、控制相关与结构冲突等问题,需采用数据前递、分支预测、流水线停顿等技术逐一解决,这也让我认识到流水线设计的复杂性与工程性。

本次实验成功实现了ADD、ADDU、ADDI、ADDIU、SLL、LW、SW、SUBU、BNE、BEQ、SLTU、HALT共12条指令的流水线执行,通过数据前递技术解决了大部分数据相关问题,借助分支预测机制降低了控制相关带来的性能损耗。

流水线技术虽能有效提升CPU的执行效率,但也引入了数据冒险、控制冒险与结构冒险等新问题。通过本次实验,我不仅掌握了流水线的基本原理与设计方法,还学会了分析与解决流水线中的各类冲突问题,对计算机系统结构的底层实现有了更深刻的认知。同时,FPGA仿真与下板测试的过程,也让我体会到硬件设计从理论到实践的转化过程,提升了工程实践能力。

\section{附件(所有程序)}

\subsection{5级指令流水线的设计程序}

\subsubsection{顶层模块 (sccomp\_dataflow.v)}
\verbatiminput{../cpupip8.srcs/sources_1/new/sccomp_dataflow.v}

\subsubsection{CPU核心模块 (cpu.v)}
\verbatiminput{../cpupip8.srcs/sources_1/new/cpu.v}

\subsubsection{定义文件 (def.v)}
\verbatiminput{../cpupip8.srcs/sources_1/new/def.v}

\subsubsection{算术逻辑单元 (alu.v)}
\verbatiminput{../cpupip8.srcs/sources_1/new/alu.v}

\subsubsection{分支判断模块 (BJudge.v)}
\verbatiminput{../cpupip8.srcs/sources_1/new/BJudge.v}

\subsubsection{PC寄存器模块 (PCreg.v)}
\verbatiminput{../cpupip8.srcs/sources_1/new/PCreg.v}

\subsubsection{数据存储器 (DMEM.v)}
\verbatiminput{../cpupip8.srcs/sources_1/new/DMEM.v}

\subsubsection{流水级间寄存器 (EX\_MEM.v)}
\verbatiminput{../cpupip8.srcs/sources_1/new/EX_MEM.v}

\subsubsection{流水级间寄存器 (ID\_EX.v)}
\verbatiminput{../cpupip8.srcs/sources_1/new/ID_EX.v}

\subsubsection{流水级间寄存器 (IF\_ID.v)}
\verbatiminput{../cpupip8.srcs/sources_1/new/IF_ID.v}

\subsubsection{指令存储器 (IMEM.v)}
\verbatiminput{../cpupip8.srcs/sources_1/new/IMEM.v}

\subsubsection{流水级间寄存器 (MEM\_WB.v)}
\verbatiminput{../cpupip8.srcs/sources_1/new/MEM_WB.v}

\subsubsection{下一PC生成模块 (NPCmaker.v)}
\verbatiminput{../cpupip8.srcs/sources_1/new/NPCmaker.v}

\subsubsection{寄存器文件 (regfile.v)}
\verbatiminput{../cpupip8.srcs/sources_1/new/regfile.v}

\subsubsection{测试平台 (246tb\_ex10\_tb.v)}
\verbatiminput{../cpupip8.srcs/sim_1/new/246tb_ex10_tb.v}

\subsubsection{自动化测试脚本 (run\_cpu\_tests.do)}
\verbatiminput{../run_cpu_tests.do}

\subsubsection{软件仿真器 (cpu\_pipelined\_simulator.cpp)}
\verbatiminput{../cpu_pipelined_simulator.cpp}

\end{document}